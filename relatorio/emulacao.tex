\section{Criação da aplicação PDF PAdES em Perl}

Para este trabalho seguiu-se a estrutura do programa original separando o corpo da aplicação em ficheiros especificos. O corpo principal do trabalho foi colocado no ficheiro \textit{signpdf\_cli.pl}, as operações sobre assinatura de ficheiros e a chave móvel através da ligação ao servidor \textit{Soap}, foram colocadas no modulo \textit{cmd\_soap\_msg.pm}, as operações referentes á comunicação com o servidor \textit{Rest} arrumaram-se num módulo chamado \textit{dss\_rest\_msg.pm}, e as variáveis \textit{APPLICATION\_ID} e \textit{DSS\_REST} ficaram no modulo \textit{cmd\_config.pm}. Adicionalmente foi criado um módulo para realizar testes e verificações de segurança sobre inputs recebidos do utilizador e dos servidores chamado \textit{verifiers.pm}.\newline


Para o corpo do programa em Perl começou-se por importar as variáveis \textit{\$APPLICATION\_ID} e \textit{DSS\_REST} do modulo \textit{signpdf\_config.pm} através das \mbox{subrotinas} \textit{get\_appid()} e \textit{get\_rest()} respetivamente. Caso a variável \textit{\$APPLICATION\_ID} não esteja definida o programa termina.

\par Em seguida verifica-se se o programa foi invocado com argumentos, caso contrário o programa também termina informando o utilizador que deve utilizar o comando \textit{signpdf\_cli.py [-h]} para obter ajuda sobre como funciona o programa. Caso tenha sido invocado com argumentos é realizado um parser dos dados recorrendo ao módulo \textit{Getopt::Long} que faz uso de flags para identificar inequivocamente cada variável recebida como parâmetro. Em seguida estes argumentos são colecionados num array cujas variáveis definidas serão verificadas fazendo uso das subrotinas pertencentes ao módulo \textit{verifiers.pm}. Adicionalmente é verificado se os dados que têm obrigatóriamente de ser fornecidos ao programa foram dados como input pelo utilizador, caso contrário o programa também termina.\newline
Caso o input passe nas verificações de segurança, é dado inicio ao processo de assinatura do pdf recorrendo á chave móvel digital.\newline
