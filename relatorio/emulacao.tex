\section{Criação da aplicação PDF PAdES em Perl}

Para este trabalho seguiu-se a estrutura do programa original separando o corpo da aplicação, que ficou no ficheiro signpdf\_cli.pl, das operações sobre ficheiros, chave móvel e ligação ao servidor SOAP, que ficaram no modulo cmd\_soap\_msg.pm, das operações realizadas por um servidor REST que se colocaram num módulo chamado dss\_rest\_msg.pm, e das variáveis APPLICATION\_ID e DSS\_REST que ficaram no modulo cmd\_config.pm. Adicionalmente foi criado um módulo para tratamento de segurança da aplicação chamado \textit{verifiers.pm}.\newline


Para o corpo do programa o Perl começa por importar as variáveis \textit{\$APPLICATION\_ID} e \textit{DSS\_REST} do modulo \textit{signpdf\_config} através das subrotinas \textit{get\_appid()} e \textit{get\_rest()} respetivamente. Caso a variável \textit{\$APPLICATION\_ID} não esteja definida o programa termina. \newline

Em seguida é verificado se o programa foi invocado com argumentos, caso contrário o programa também termina informando o utilizador que deve utilizar o comando \textit{signpdf\_cli.py [-h]} para obter ajuda sobre como funciona o programa. Caso tenha sido invocado com argumentos é realizado um parser dos dados recorrendo ao módulo \textit{Getopt::Long} que faz uso de flags para identificar inequivocamente cada variável recebida como parâmetro. Em seguida estes argumentos são colecionados num array cujas variáveis definidas serão verificadas fazendo uso das subrotinas pertencentes ao módulo \textit{verifiers.pm}.\newline
Caso o input passe nas verificações de segurança, é dado inicio ao processo de assinatura do pdf recorrendo á chave móvel digital.\newline
Convêm referir que, ao contrário do que foi realizado no ficheiro original \textit{signpdf\_cli.py} as verificações de segurança referentes ao código das mensagens recebidas do servidor foram realizados no módulo \textit{cmd\_soap\_msg.pm}.

\subsection{Módulos}

Para instalar os módulos necessários pode-se utilizar uma ferramenta chamada \textit{cpanm}. Pode-se descarregar esta ferramenta para linux com o comando de terminal \textit{sudo apt install cpanminus}.\newline
Após instalar a ferramenta deve-se garantir que se têm os seguintes módulos instalados\footnote{Nota: Para descarregar os módulos use no terminal o comando cpanm install 'nome do módulo'}:

\begin{enumerate}
	\item Crypt::OpenSSL::X509
	\item DateTime
	\item Digest::SHA
	\item MIME::Base64
	\item Getopt::Long
	\item POSIX
	\item List::MoreUtils
	\item Carp
	\item FindBin
	\item IO::Prompt
	\item REST::Client
	\item XML::Compile::WSDL11
	\item XML::Compile::SOAP11
	\item XML::Compile::Transport::SOAPHTTP
	\item Encode
	\item Bit::Vector
	\item HTTP::Request
	\item HTTP::Parser
	\item Log::Log4perl
	\item LWP::ConsoleLogger
	\item strict
\end{enumerate}